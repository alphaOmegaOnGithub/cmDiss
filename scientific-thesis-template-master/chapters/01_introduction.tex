Large scale organisations have to deal with continuously increasing amounts of structured, semi-structured or unstructured information due to the progress in digitalization.
Especially semi-structured data like business correspondences and multi media content have seen a rapid growth driven by trends like the paperless office and the large adaption of mobile devices.
Additionally documents that support or result from essential business processes need to be stored in an audit-compliant manner for various regulatory reasons.
To utilize the relevant information contained in documents, emails or media files throughout their whole life cycle, organizations rely on Enterprise Content Management Systems.
Those systems are typically deployed as a monolithic applications on an on-premise infrastructure with a long-running update cycle which oftentimes requires a dedicated team within the IT department.
To deliver the value creation that ECM systems generate to smaller enterprises which can not afford a large IT-Team or are unable to acquire the necessary talent various vendors launched Enterprise Content Management on Cloud infrastructures.
This offer could only be facilitated through the continuous advancement of virtualization, containerization and orchestration technologies as well as the rapid growth of network bandwidth.
\\
\\
To further optimize the utilization of the infrastructure of ECM on Cloud components based on truly occurring workloads the following thesis aims to integrate a decomposed and containerized ECM application into a cluster running on a cloud environment orchestrated by Kubernetes.
The conducted investigation focused on finding a feasible system topology that provides a stable and reliable environment for organizations to store their business critical data.