This thesis demonstrates that the approach by Shao~\cite{shao} can be operated properly on a \textit{Docker} based container environment but it was not possible to apply a one to one porting into a Kubernetes cluster.
The cluster needs to be further enhanced to handle stateful database services in an automated manner by leveraging Kubernetes technology.
\\
\\
This work evaluates two approaches of operating stateful applications inside Kubernetes.
Based on this investigation the first concept is proposed and implemented in the form of a prototype.
While conducting reliability tests it turned out that the developed solutions of running the \textit{Data Catalog} and \textit{Object Catalog} of the ECM system could not guarantee a stable and reliable operation inside the cluster. 
The objective of managing stateful database applications inside a Kubernetes cluster poses as a serious challenge and requires a much more complex deployment design.
This is mainly because Kubernetes was designed to autonomously manage stateless workloads.
Therefore the stateful components were removed and managed on an external infrastructure.
In the end the effort was split in two phases: primarily focus on implementing the stateless components and secondarily further improve the initial approach to handle stateful components.
Given the time constraints for this thesis and the complexity of the task the stateful databases services are left on the external docker environment.
Additionally the stateless applications which stayed inside the cluster required the creation of a new Docker image which eliminated the need of external data sources mounted into the \textit{Pods}.
\\
\\
Future work might improve the developed prototype through investigating the areas of load balancing, dynamic scaling and security in Kubernetes clusters.