%
% Pakete um UFT8 Zeichnens�tze verwenden zu k�nnen und die dazu passenden Schriften.
\usepackage[utf8]{inputenc}
\usepackage[T1]{fontenc}

\usepackage{color}
\usepackage{xcolor}

\usepackage[a4paper]{geometry}
\geometry{left=2.0cm,right=3.5cm,top=3.5cm,bottom=4.0cm}

%
% eigene Quotation Umgebung
%
\setlength{\marginparwidth}{2cm}
\newcommand{\quotepaper}[2]{\marginpar{\tiny \textbf{\color{red}QUOTE:
#1}}{\color{red}#2}}

%
% wirklich h�bsche Kapitel�berschriften!
% alternativen sind: Conny, Sonny, Rejne, Lenny, Glenn, Bjornstrup 
%
\usepackage[Bjarne]{fncychap}
\ChTitleVar{\raggedleft\huge\rmfamily\bfseries}

% zur Erzeugung h�bscher Schmuckinitialen
\usepackage{lettrine}
% zum setzen der Farbe der Schmuckinitialen auf dunkles grau
\renewcommand{\LettrineFontHook}{\color[gray]{0.35}}

% Packe f�r den Lebenslauf
\usepackage[]{currvita}

\usepackage{epic}

%
% mathematische symbole aus dem AMS Paket.
%
\usepackage{amsmath}
%\usepackage{amssymb}

\usepackage{program}
\usepackage{listing}
\usepackage{newlfont}

%
% ein paar Definitionen
%
\usepackage{framed}
\usepackage{soul}
\definecolor{shadedarkcolor}{gray}{.9}
\definecolor{shadelightcolor}{gray}{.97}
\definecolor{shadedarkercolor}{gray}{.25}

\definecolor{ttcolor}{gray}{.1}

%
% Definition f�r schraffierte Box - dunkel
%
\newenvironment{shadeddark}{
  \def\FrameCommand{\fboxsep=\FrameSep \colorbox{shadedarkcolor}}
  \MakeFramed {\advance\hsize-1\width\everypar{\parshape 1 0.0cm \textwidth}\FrameRestore}}
{\endMakeFramed}

%
% Definition f�r schraffierte Box - hell
%
\newenvironment{shadedlight}{
  \def\FrameCommand{\fboxsep=\FrameSep \colorbox{shadelightcolor}}
  \MakeFramed {\advance\hsize-1\width\everypar{\parshape 1 0.0cm \textwidth}\FrameRestore}}
{\endMakeFramed}

\definecolor{quotescolor}{gray}{.5}
\newcommand*\openquote{\makebox(25,-22){\scalebox{5}{\textcolor{quotescolor}{``}}}}
\newcommand*\closequote{\makebox(25,-22){\scalebox{5}{\textcolor{quotescolor}{''}}}}

\newcommand{\abs}[1]{\ensuremath{\left\vert#1\right\vert}}

% Literaturverzeichnis nach deutschem Standard
\usepackage{bibgerm}

%
% Paket f�r �bersetzungen ins Deutsche
%
%\usepackage[english, ngerman, german]{babel}
\usepackage[ngerman, german, english]{babel}

%
% Paket f�r Quotes
%
\usepackage[babel,german=quotes,english=american]{csquotes}

% Den �bersetzer setzen
%\usepackage[ngerman]{translator}

\usepackage{url}

%
% Paket f�r Links innerhalb des PDF Dokuments
%
\definecolor{LinkColor}{rgb}{0,0,0}
\usepackage[%
	pdftitle={Titel der Arbeit},% Titel der Diplomarbeit
	pdfauthor={Vorname Nachname},% Autor(en)
	pdfcreator={LaTeX, LaTeX with hyperref and KOMA-Script},% Genutzte Programme
	pdfsubject={Genehmigte Abhandlung zur Erlangung der Wuerde eines Doktors der Naturwissenschaften (Dr. rer. nat)}, % Betreff
	pdfkeywords={Liste, von, Keywords}, % Keywords halt :-)	
	dvipdfm,	
	breaklinks=true]{hyperref} 
\hypersetup{colorlinks=true,% Definition der Links im PDF File
	linkcolor=LinkColor,%
	citecolor=LinkColor,%
	filecolor=LinkColor,%
	menucolor=LinkColor,%
	pagecolor=LinkColor,%
	urlcolor=LinkColor,
	breaklinks=true}
%\usepackage{breakurl}
%\usepackage{hypdvips}

% fuer Abk�rzungsverzeichnis
% ------------------- Abk�rzungsverzeichnis: ----------------------------
\usepackage[
%nonumberlist, %keine Seitenzahlen anzeigen
acronym,      %ein Abkürzungsverzeichnis erstellen
shortcuts,    %
toc %,          %Einträge im Inhaltsverzeichnis
%section	%im Inhaltsverzeichnis auf section-Ebene erscheinen
]
{glossaries}
%\deftranslation{Acronyms}{List of Abbreviations}

%Ein eigenes Symbolverzeichnis erstellen
\newglossary[slg]{symbolslist}{syi}{syg}{Symbolverzeichnis}

%Den Punkt am Ende jeder Beschreibung deaktivieren
\renewcommand*{\glspostdescription}{}

\newglossarystyle{myacrstyle}{%
  \renewenvironment{theglossary}%
     {\begin{longtable}{llp{\glsdescwidth}}}%
     {\end{longtable}}%
  \renewcommand*{\glossaryheader}{}%
  \renewcommand*{\glsgroupheading}[1]{}%
  \renewcommand*{\glossaryentryfield}[5]{%
    \glstarget{##1}{##2} & ##3\glspostdescription\space ##5\\}%
  \renewcommand*{\glossarysubentryfield}[6]{%
     & \glstarget{##2}{\strut}##4\glspostdescription\space ##6\\}%
  \renewcommand*{\glsgroupskip}{ & \\}%
}

%Glossar-Befehle anschalten
\makeglossaries

%Diese Befehle sortieren die Eintr�ge in den
%einzelnen Listen:
%makeindex -s datei.ist -t datei.alg -o datei.acr datei.acn
%makeindex -s datei.ist -t datei.glg -o datei.gls datei.glo
%makeindex -s datei.ist -t datei.slg -o datei.syi datei.syg

%\makenomenclature
%------------------------------------------------------------------------

% fuer Stichwortverzeichnis
\usepackage{makeidx}
%\usepackage{index}
\makeindex

%
% special math environment
%
\usepackage{latexsym}

\usepackage{subfigure}
\usepackage{dblfloatfix}
%\usepackage{booktabs}
\usepackage{multirow}
%\usepackage{fancybox}
\usepackage{verbatim}
%\usepackage{todonotes}
\usepackage{booktabs}

\usepackage{tabularx}
\usepackage{rotating}
\usepackage{threeparttable}
\usepackage{colortbl}

\usepackage{lscape}

\usepackage{afterpage}

% 
% f�r h�nsche Listings - START
% 
\usepackage{listings}
%\lstloadlanguages{TeX, XML}
\usepackage{courier}

\lstdefinelanguage{JavaScript}
  {morekeywords={break, else, new, var, case, finally, return, void,
catch, for, switch, while, continue, function, this, with, default, if,
throw, delete, in, try, do, instanceof, typeof, alert,
abstract, enum, int, short, boolean, export, interface, static, byte,
extends, long, super, char, final, float, const, private,
debugger, implements, protected, volatile, double, import, public},
   sensitive,
   morecomment=[l]//,
   morecomment=[s]{/*}{*/},
   morestring=[b]",
   morestring=[b]',
  }[keywords,comments,strings]
  
\lstdefinelanguage{WebIDL}
  {morekeywords={interface, sequence, optional, readonly, attribute,
boolean, long, getter, exception, const, unsigned, void, short},
   sensitive,
   morecomment=[l]//,
   morecomment=[s]{/*}{*/},
  }[keywords,comments]

\lstset{
		language=XML,
%         basicstyle=\footnotesize\ttfamily, % Standardschrift
         basicstyle=\small\sffamily, % Standardschrift
%         basicstyle=\small\ttfamily, % Standardschrift
         numbers=left,               % Ort der Zeilennummern
         numberstyle=\tiny,          % Stil der Zeilennummern
         columns=fullflexible,
         %stepnumber=2,               % Abstand zwischen den Zeilennummern
         numbersep=10pt,              % Abstand der Nummern zum Text
         backgroundcolor=\color{shadelightcolor},      % choose the background color. You must add \usepackage{color}
         tabsize=2,                  % Groesse von Tabs
         extendedchars=true,         %
         captionpos=b,                   % sets the caption-position to bottom
         breaklines=true,            % Zeilen werden Umgebrochen
         keywordstyle=\textbf, %\color{red},
    	frame=tb,
    	columns=fullflexible,
 %        keywordstyle=[1]\textbf,    % Stil der Keywords
 %        keywordstyle=[2]\textbf,    %
 %        keywordstyle=[3]\textbf,    %
 %        keywordstyle=[4]\textbf,   \sqrt{\sqrt{}} %
%         stringstyle=\color{shadedarkercolor}\ttfamily, % Farbe der String
         stringstyle=\color{shadedarkercolor}\sffamily, % Farbe der String
         showspaces=false,           % Leerzeichen anzeigen ?
         showtabs=false,             % Tabs anzeigen ?
         xleftmargin=17pt,
         framexleftmargin=17pt,
         framexrightmargin=0pt,
         framexbottommargin=4pt,
         rulesepcolor=\color{shadelightcolor},
         %backgroundcolor=\color{lightgray},
         showstringspaces=false,      % Leerzeichen in Strings anzeigen ?        
         abovecaptionskip=\abovecaptionskip,
         belowcaptionskip=\belowcaptionskip,
         aboveskip=1.5\floatsep,
         floatplacement=hbt!
         %xleftmargin=0cm,
         %xrightmargin=0cm
 }

\usepackage{caption}
\captionsetup{labelfont=bf}

% definiert die Widmungsseite
\usepackage{styles/dedication}

% nur vorerst zum Testen
\usepackage{blindtext}

%
% Paket zum Erweitern der Tabelleneigenschaften
%
\usepackage{array}

%
% Paket für schönere Tabellen
%
\usepackage{booktabs}

%
% Paket um Grafiken einbetten zu können
%
\usepackage{lscape,graphicx}
\usepackage{subfigure}
\usepackage{dblfloatfix}
%\usepackage{graphicx}

%
% Definitionen wir die Abbildungen etc. zu hei�en haben.
%
\def\figurename{Figure }
\def\equationname{Equation }
\def\tablename{Table }
\def\listingname{Listing }

\def\sectionname{Section }
\def\chaptername{Chapter }

%
% Zeilenumbruch bei Bildbeschreibungen.
%
\setcapindent{1em}

%
%--------- Kann nur verwendet werden, wenn "fancyhdr" ausgeschaltet ist!!! -----
%
% KOMA-Script-Paket f�r schicke Kopf- und Fu�zeilen
\usepackage[automark]{scrpage2}
\usepackage{floatflt}
\usepackage{lscape}
\usepackage{rotating}
\usepackage{textcomp}

\usepackage{scrtime}
\usepackage{scrdate}

% Kopf- und Fu�zeile
\clearscrheadfoot	 % alten Standardkram raus
\ohead[\pagemark]{\pagemark}	% oben rechts Seitenzahl laut Richtlinie
%\chead[]{}
\ihead{\headmark}	 % automatischen Kapitelnamen rein

%\ifoot{\author}
%\cfoot{untenmitte}
\ofoot{\small \todaysname, \today -- \thistime[:]}

% möglicherweise Zeilenabstand in Kopfzeile auf 1 setzen
%\addtokomafont{pagehead}{\linespread{1}\selectfont}
%\setkomafont{pagehead}{\linespread{1}}
\pagestyle{scrheadings}
%
%--------- Kann nur verwendet werden, wenn "fancyhdr" ausgeschaltet ist!!! -----
%

% serifenlose Schrift als Default verwenden
%\renewcommand{\familydefault}{\sfdefault}

% Spezielle Schrift im Koma-Script setzen. --> sollte nicht n�tig sein
%\setkomafont{sectioning}{\sffamily\bfseries}
%\setkomafont{captionlabel}{\sffamily\bfseries}
%\setkomafont{pagehead}{\sffamily\itshape}
%\setkomafont{pagenumber}{\sffamily\itshape}
%\setkomafont{descriptionlabel}{\sffamily\bfseries}


%\newcommand{\berfont}{
%	\fontencoding{T1}
%	\fontfamily{beramono}
%	\fontseries{m}
%	\fontshape{it}
%	\fontsize{12}{15}
%	\selectfont
%}

%\setkomafont{sectioning}{\berfont\bfseries}
%\setkomafont{captionlabel}{\berfont\bfseries}
%\setkomafont{pagehead}{\berfont\itshape}
%\setkomafont{pagenumber}{\berfont\itshape}
%\setkomafont{descriptionlabel}{\berfont\bfseries}

\addtokomafont{sectioning}{\rmfamily}

%Fuer englische Texte sind serifenhafte Ueberschriften gut. Deshalb hier der Befehl zum Aktivieren von serifenhaften Ueberschriften 
\setkomafont{disposition}{\bfseries\rmfamily}

%\setkomafont{chapterentrypagenumber}{\sffamily\bfseries}
%\usekomafont{disposition}
%\usepackage{palatino}
%\usepackage{times}
%\usepackage{utopia}
\usepackage{beramono}
%\usepackage{libertine}
%\typearea{calc}

%% Only if the base font of the document is to be sans serif
%\renewcommand*\familydefault{\sfdefault}
%\setkomafont{disposition}{\sffamily\bfseries}
%\addtokomafont{sectioning}{\sffamily\bfseries}
% %
% Otherwise use this setting
\setkomafont{disposition}{\rmfamily\bfseries}
\addtokomafont{sectioning}{\rmfamily\bfseries}

%\renewcommand{\rmdefault}{ppl}
%\renewcommand{\sfdefault}{phv}
%\renewcommand{\ttdefault}{pcr}

%\linespread {1.0}
%\linespread {1.05}
\linespread {1.10}
%\linespread {1.15}

%\usepackage{ae} % Schöne Schriften für PDF-Dateien

\usepackage{microtype}

\setlength{\parindent}{1em}
\setlength{\parskip}{0.35em}

%
% Paket um Textteile drehen zu können
%
\usepackage{rotating}

% ---------------------------------------------------------------------------
%

%\usepackage{cslibib}
\usepackage[sectcntreset]{bibtopic}

%
% Literaturverzeichnis-Stil
%
\usepackage[%
	%round,	%(default) for round parentheses;
	square,	% for square brackets;
	%curly,	% for curly braces;
	%angle,	% for angle brackets;
	%colon,	% (default) to separate multiple citations with colons;
	comma,	% to use commas as separaters;
	%authoryear,% (default) for author-year citations;
	numbers,	% for numerical citations;
	%super,	% for superscripted numerical citations, as in Nature;
	sort,		% orders multiple citations into the sequence in which they appear in the list of references;
	sort&compress,    % as sort but in addition multiple numerical citations
                  % are compressed if possible (as 3-6, 15);
	%longnamesfirst,  % makes the first citation of any reference the equivalent of
                  % the starred variant (full author list) and subsequent citations
                  %normal (abbreviated list);
	%sectionbib,      % redefines \thebibliography to issue \section* instead of \chapter*;
                  % valid only for classes with a \chapter command;
                  % to be used with the chapterbib package;
	%nonamebreak,     % keeps all the authors names in a citation on one line;
                  %causes overfull hboxes but helps with some hyperref problems.
	%authoryear, 	% to enable author-year citation support
]{natbib}
%\bibliographystyle{alphabin} % alternative styles: natdin, plaindin
%\bibliographystyle{plainnat} % alternative styles: natdin, plaindin
%\bibliographystyle{natdin} % alternative styles: natdin, plaindin
%\bibliographystyle{abbrvdin}

%\setlength\bibhang{3em}

% Strukturiertiefe bis subsubsection{} möglich
%
\setcounter{secnumdepth}{3}

%
% Dargestellte Strukturiertiefe im Inhaltsverzeichnis
%
%\setcounter{tocdepth}{3}
\setcounter{tocdepth}{2}

%
%Zeilenabstand wird um den Faktor 1.5 verändert
%
%\renewcommand{\baselinestretch}{1.25}

% Disable single lines at the start of a paragraph (Schusterjungen)
\clubpenalty = 10000

% Disable single lines at the end of a paragraph (Hurenkinder)
\widowpenalty = 10000 
\displaywidowpenalty = 10000

%\renewcommand{\abstractname}{Executive Summary}

% Seitengroessen - Gegen Schusterjungen und Hurenkinder... 
\newcommand{\largepage}{\enlargethispage{\baselineskip}} 
\newcommand{\shortpage}{\enlargethispage{-\baselineskip}} 

\setlength{\emergencystretch}{3em} % Silbentrennung reduzieren durch mehr frei Raum zwischen den Worten
