% -*- root: ../thesis.tex -*-
% Thesis Abstract -----------------------------------------------------
\selectlanguage{german}

\begin{zusammenfassung}        %this creates the heading for the abstract page
\addcontentsline{toc}{chapter}{Conclusion and outlook (Zusammenfassung)}
 In our paper we showed a possible evolutionary path  of a n ECM systems, outlined their core functionality and the services they provide to businesses. We discussed a set of representative use case scenarios and defined the workload model from where we derived the blueprint of ECM systems their underlying architecture design. From our analysis we found that the repository framework consist of a globally distributed, heterogeneous infrastructure build from commodity hardware components deployed on- or off-premise or a combination of both. We also learned that the key non-functional characteristics of modern ECM systems are: flexibility, configurability and scale at an affordable cost. Current trends show these goals can be achieved by means of abstraction, virtualization and containerization thus allowing an indirection level to be introduced for creating a single logical ECM entity that is geographically distributed in nature.
 Many decision makers in larger Enterprises have mandated the restructure of their IT-Environment such to cut cost and increase productivity. We have identified the the key areas of outdated, inflexible and slow IT-Processes. Over the years these homegrown processes have become a major show stopper when it comes to provided more dynamic, flexible and fast resource allocation services. The established application deployment processes no longer satisfy demand in an ‘Everything as a Service’ (EaaS) world. The way out of this dilemma is the adoption of ‘Cloud’-technology. The latter has industrialized the delivery of managed services by employing a new consumption model and thus showing a clear way out into a more efficient IT-Infrastructure. 
\end{zusammenfassung}
\ifCLASSINFOlangDE
\selectlanguage{german}
\else
\selectlanguage{english}
\fi
% ---------------------------------------------------------------------- 
