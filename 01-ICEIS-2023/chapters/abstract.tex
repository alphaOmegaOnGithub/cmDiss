\abstract{
Information is a strategic asset to enterprises and subject to Information Governance (IG) as mandated by corporate and regulatory compliance. Overall governance goals are the management and control of business relevant data such to minimize legal risks and reduce operational cost. Recent surveys show that around 40\% of companies have insufficient IG practices in place and are exposed to higher compliance risks. Where there exists an IG Strategy, its implementation is typically homegrown, difficult to integrate and error prone. Our investigation shows that a major impediment to implementation and interoperability is the lack of a common language. One that defines what information governance consists of in technical and operational terms. In addition, the many existing frameworks in this domain make the exploitation of the available knowledge and definition of standardized IG specific services very difficult, often leading to costly one-off implementations. A solution to this problem is the availability of a common and unambiguous domain vocabulary as a pre-requisite to a commonly accepted ontology on information governance. This paper suggests an ontology-based framework (IGONTO) that supports the creation of a knowledge store that facilitates access of domain knowledge through semantic search.
}
