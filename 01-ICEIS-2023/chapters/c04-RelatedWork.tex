\section{\uppercase{Related work}}
\label{sec:Relatedwork}
\cite{DeStefano2016} DeStefano and Tao inspired this work using a similar approach to address a different data governance problem. \cite{Casonato2011} (Casonato, 2011) elaborates on the relationship between Enterprise Information Management and Information Governance showing that IG is an extension of EIM. \cite{Leibtag2014}(Leibtag, 2014) explains the necessity of governance in enterprises. Proença, et al. re-elaborates on a maturity model for information governance IGMM) \cite{Proenca2018}. Cheng, Li and Gao Liu looked at the governance maturity model and its implementation investigating the impact of storing and managing data in the context of cloud \cite{Cheng2017}. Yamada and Peran suggests a governance framework for enterprise analytics and data \cite{Yamada2017}. Al-Ruithe and Hameed compared data governance on cloud versus non-cloud environments \cite{ALRuithe2018}. Abraham and Schneider reviewed data governance conceptual frameworks suggesting a research agenda on white spaces in the IG research areas \cite{Abraham2019}. Mahanti introduced the topics of Data, Data Governance, and Data Management \cite{Mahanti2021}. Also the many ISO international standards related to information governance like ISO-14271, ISO-15489 \cite{ISO15489} and the ISO 27000 family of standards related to security \cite{ISO27000}. Little work was found related on how to implement an IG framework from concept to actual implementation. None using pre-defined cloud services. Nor have we found the definition of a IG semantic schema that utilizes existing implementation knowledge.Mubarkoot and Altmann published some on software compliance in the context of information governance \cite{Mubarkoot2021}. We found also valuable concept work in W3C FIBO ontology from the EDM Council \cite{FIBO2020}.
\vfill