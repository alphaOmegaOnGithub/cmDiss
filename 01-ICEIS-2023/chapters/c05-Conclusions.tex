\section{\uppercase{Conclusions and Outlook}}
\label{sec:conclusion}
This paper describes our ontology-based knowledge management model on Information Governance and presented the IGONTO framework. Major effort in our work was to collect domain knowledge from available IG frameworks. We evaluated their taxonomies, vocabularies, glossaries and best practices. Then re-engineered the taxonomy, enhanced the terms that represent core IG concepts, defined the semantic schema (IGO) and created a graph store (IGG). Available solutions components were stored in the IGREPO repository as artefacts. Our work provides a way for interested parties, to use the framework as a system that makes domain knowledge and experience explicit and accessible through semantic queries. Result sets produced consist of generalized service concepts including descriptions of their functional, non-functional and operational capabilities and a list of IG program requirements they satisfy. We think of it as an accelerator for building information governance solutions composed of pre-defined IG services artefacts offered on clouds discovered through an IG-graph store. As an outlook on future work, we are currently refining the IGONTO prototype and working on a DSL for IG to extend TOSCA such to provide a stronger bridge between concept, design and implementation.