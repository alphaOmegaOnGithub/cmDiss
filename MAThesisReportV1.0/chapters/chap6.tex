\chapter{CONCLUSION AND FUTURE WORK}

\paragraph\ 
Managing the large amount of information generated every day is an important task. Information Governance ensures that the data is handled appropriately according to acceptable standards. Retrieving information from stored data is necessary in order to be able to better manage it. Hence, in this thesis the leading cloud service providers that provide information retrieval as a service to their customers were researched. As data privacy and security are extremely critical, the service providers have security measures in place to be compliant with the \acs{GDPR} regulations. Few of them have their audit reports and compliance certificates publicly available so the customers are well informed of the data protection methods followed by these companies. Although there weren't any \acs{GDPR} certificates available to ensure the company's adherence to \acs{GDPR}, the author was able to find the ISO 27018 certificates which ensure the protection of personal data on the cloud. The various cloud service providers offered different features of information extraction and it was possible to test a few of these features on the selected datasets. As the testing was done with 'free-tier' or 'basic' access, there was no possibility to experiment much with large datasets. Due to these limitations like time and access to the services, testing of all the researched services providers' \acs{IR} capabilities on all the datasets was not possible. However, this thesis provides a good starting point for beginners, as it provides initial research documentation for \acs{IR} services provided by the cloud service providers in detail and gives a tabular comparison of what features are available in each of them (which were tested during this thesis duration). Testing other \acs{IR} features offered by other cloud service providers on larger datasets can be done as an extension of this thesis work.  