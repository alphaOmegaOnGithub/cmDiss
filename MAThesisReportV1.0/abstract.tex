\chapter*{ABSTRACT}%

\addcontentsline{toc}{chapter}{ABSTRACT}
\paragraph\

Information is a vital part of today's world. Collecting the information, storing it, processing it, and deleting are tasks that are to be done with great responsibility. Information governance summarises these tasks and ensures the privacy and security of the data. Information Retrieval is a crucial step in information governance as it is important to know what the data consists of, to be able to determine how or where to store and process it. And due to this huge demand for information retrieval tasks, many companies offer it as a service on cloud platforms for users. This thesis researches the leading cloud service providers that provide information retrieval as a service to their customers. The features are then tested for a few of the researched service providers. A comparison is made of the cloud service providers on a few of the metrics and results are tabulated for the users to make an informed decision. Compliance with the \ac{GDPR} guidelines is also researched and the measures are in place by each of them to handle data transfer between different countries and storage of users' data securely. The compliance certificates like \acs{ISO} 27018 and links to other relevant reports are researched. This thesis would help users skip initial research steps and provide the required information in a well-structured manner with directions to relevant resources. The user can then make informed decisions on what features and which cloud provider is better suited for their specific requirements.\\

\textbf{Keywords:} Information Governance, Information Retrieval, Machine Learning, Artificial Intelligence, \acs{GDPR}.
\pagenumbering{roman}
